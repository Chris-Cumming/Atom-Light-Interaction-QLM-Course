%--------------------
% Packages
% -------------------
\documentclass[11pt,a4paper]{article}
\usepackage[utf8x]{inputenc}
\usepackage[T1]{fontenc}
%\usepackage{gentium}


\usepackage{gensymb} %Initially only used for degrees symbol


\usepackage[pdftex]{graphicx} % Required for including pictures
\usepackage[pdftex,linkcolor=black,pdfborder={0 0 0}]{hyperref} % Format links for pdf
\usepackage{calc} % To reset the counter in the document after title page
\usepackage{enumitem} % Includes lists

\frenchspacing % No double spacing between sentences
\linespread{1.2} % Set linespace
\usepackage[a4paper, lmargin=0.1666\paperwidth, rmargin=0.1666\paperwidth, tmargin=0.1111\paperheight, bmargin=0.1111\paperheight]{geometry} %margins
%\usepackage{parskip}

\usepackage[all]{nowidow} % Tries to remove widows
\usepackage[protrusion=true,expansion=true]{microtype} % Improves typography, load after fontpackage is selected



%-----------------------
% Set pdf information and add title, fill in the fields
%-----------------------
\hypersetup{ 	
pdfsubject = {},
pdftitle = {},
pdfauthor = {}
}

%-----------------------
% Begin document
%-----------------------
\begin{document} %All text i dokumentet hamnar mellan dessa taggar, allt ovanför är formatering av dokumentet

\title{Atom-Light Interactions Notes}

\maketitle

\section{Questions}
\begin{itemize}
    \item 
\end{itemize}

\section{Preliminaries}

\subsection{Numerical Solutions of coupled ODEs}

Quite often models in atom-light interactions can be decomposed into solving coupled ODEs. Generally these will be solved as IVP. This means solving equations of the form:

\begin{equation}
    \frac{d \underline{u}}{dt} = \underline{f}(\underline{u}, t),
    \label{CoupledODEs}
\end{equation}

\noindent where we wish to solve for $\underline{u}(t_0 + t)$ given $\underline{u}(t_o) = \underline{u}_0$. An example of this is solving the equation of motion for a simple pendulum. This has an equation of motion for the angular degree of freedom as:

\begin{equation}
    \frac{d^2 \theta}{dt^2} = -\sin(\theta).
    \label{PendulumEOM}
\end{equation}

\noindent This 2nd order ODE can be decomposed into two coupled first order ODEs:

\begin{equation}
    \frac{d \theta}{dt} = \dot{\theta}; ~ \frac{d \dot{\theta}}{dt} = -\sin(\theta),
    \label{CoupledPendulumODEs}
\end{equation}

\noindent which matches our general definition, Eq~\ref{CoupledODEs}, if $\underline{u} = (\theta, \dot{\theta})$.

\subsection{Matrix Representation of QM}

\section{Atoms in Zero Field}

\section{Atoms in Static Field}

\section{Atoms in Dynamic Fields}

\section{Semi-Classical Atom-Light Interactions}

\section{Spontaneous Decay and Optical Bloch Equations}

\section{Atomic Ensembles and Light}

\section{Three-Level Systems}

\bibliography{refs}

\bibliographystyle{h-physrev5}


\end{document}
